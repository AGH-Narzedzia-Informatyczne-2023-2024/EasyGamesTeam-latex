

\newpage
\section{Harelikau Waladar}
\label{sec:text_vharelikau}

\subsection{matematyczne wyrazenie}
    \begin{equation}
              \omega_c = \frac{1}{RC}\\
    \end{equation}

\subsection{text}
    \begin{flushleft}
         \textbf{Nocoloty} mają maskujące, miękkie ubjgjgpierzenie, małe dzioby z szeroką paszczą i baadsdasadsrdzo duże oczy. Obie płcie wyglądają podobnie. Osiągają wielkość \underline{do 50 cm}. Nie budują gniazd – samica składa 1–2 jaja w małych zagłębieniach w pniach lub gałęziach. Wysiadywanie trwa ok. 33 dni, \textit{pisklę opuszcza miejsce} wylęgu po 44–50 dniach. \par Tagschläfer haben einen langgestreckten, zylindrischen Körper, einen großen Kopf, große Augen und einen kurzen Hals.(see Image~\ref{fig:nocolot}) Der Schnabel ist zart und an der Spitze nach unten gebogen. Geöffnet ist der Schnabelspalt aber sehr weit und breit. Das Gefieder ist camouflageartig und überwiegend braun und grau mit wenigen weißen oder schwarzen Federn. Die Flügel und der Schwanz sind lang. Beine und Zehen sind kurz. Die Geschlechter unterscheiden sich kaum.[1]
        wprowadzanie zmian GIT valadarTagschläfer haben einen langgestreckten, zylindrischen Körper, einen großen Kopf, große Augen und einen kurzen Hals.(see Image~\ref{fig:nocolot}) Der Schnabel ist zart und an der Spitze nach unten gebogen. Geöffnet ist der Schnabelspalt aber sehr weit und breit. Das Gefieder ist camouflageartig und überwiegend braun und grau mit wenigen weißen oder schwarzen Federn. Die Flügel und der Schwanz sind lang. Beine und Zehen sind kurz. Die Geschlechter unterscheiden sich kaum.[1]
        wprowadzanie zmian GIT valadar

             \centering
             \includegraphics{pictures/nocolot.jpg}
             \label{fig:nocolot}
         
\end{flushleft}





\subsection{list Sample}
Sample List:
\begin{enumerate}
     \item sampleItem1
      \item sampleItem2
       \item sampleItem3
        \item sampleItem4
        \label{tab:sampleTable}
\end{enumerate}
Another One lists
\begin{itemize}
  \item Pierwy punkt
  \item Drugi punkt
  \item Trzeci punkt
\end{itemize}
\newpage

\subsection{Table}

\begin{table}[h]
\centering
\begin{tabular}{lllll}
\cline{1-3}
\multicolumn{1}{|l|}{text1} & \multicolumn{1}{l|}{text2} & \multicolumn{1}{l|}{text3} &                      &  \\ \cline{1-3}
\multicolumn{1}{|l|}{12}    & \multicolumn{1}{l|}{22}    & \multicolumn{1}{l|}{32}    &                      &  \\ \cline{1-3}
\multicolumn{1}{|l|}{13}    & \multicolumn{1}{l|}{23}    & \multicolumn{1}{l|}{33}    & \multicolumn{1}{c}{} &  \\ \cline{1-3}
                            &                            &                            &                      & 
\end{tabular}
        \caption{table Sample}
 \label{tab::powers}
\end{table}
 

\newpage
